\documentclass[12pt, letterpaper]{article}
\usepackage{
    graphicx,
    chemmacros,
    tikz,
    pgfplots,
    pgfplotstable,
    filecontents,
    datatool
}

\title{Determination of the heat of hydration of bivalent copper sulphate pentahydrate}
\author{Afanasev S.M., Nesterov I.D.}
\date{}

\providecommand{\keywords} {
    \small	
    \textbf{\textit{Keywords:}}
} 

\begin{document}
    \maketitle
    \keywords{
        heat of formation,
        crystalline hydrate,
        copper sulphate}
    \tableofcontents
    \newpage

    \section{Introduction}
        \textbf{\textit{Purpose of work:}} to determine the heat of formation of crystalline hydrate \ch{CuSO4*}5\ch{H2O}. \\ \\
        The heat of formation of crystalline hydrate \ch{CuSO4*}5\ch{H2O} will correspond to
        the reaction of addition of 5 moles of water to 1 mole of anhydrous salt according to the equation:

        \begin{equation}
            \ch{CuSO4\sld{}} + 5 {\ch{H2O}\lqd{}} = \ch{CuSO4*}5\ch{H2O\sld{},} \hspace{10 mm} \enthalpy*(hyd){}
        \end{equation}

        The thermal effect cannot be determined experimentally,
        because the formation of a new phase from liquid to solid phase is very slow and incomplete.
        Theoretical calculation of the thermal effect is carried out on the basis of Hess's law
        on the basis of the following cycle:















    \section{Methods}
        do this, then this, and so on


















    \section{Results}
        Impressive Results
        
        \subsection{First experiment}  
            \centering  
            \begin{tikzpicture}
                \begin{axis}[
                        ylabel=$T$,
                        xlabel=$t$,
                        xmin=0, xmax=330,
                        ymin=21, ymax=23.2,
                        xtick={0,20,...,300},
                        xticklabel style={rotate=45, anchor=east},
                        ytick={21,21.2,...,23},
                        axis lines=middle,
                        grid=both
                    ]
                    \addplot [only marks] table [col sep=comma] {../data/raw/experiment1.csv};
                \end{axis}
            \end{tikzpicture}
        


        \subsection{Second experiment}
            \centering    
            \begin{tikzpicture}
                \begin{axis}[
                        ylabel=$T$,
                        xlabel=$t$,
                        xmin=0, xmax=330,
                        ymin=22, ymax=23.5,
                        xtick={0,20,...,300},
                        xticklabel style={rotate=45, anchor=east},
                        ytick={22,22.2,...,23.4},
                        axis lines=middle,
                        grid=both
                    ]
                    \addplot [only marks] table [col sep=comma] {../data/raw/experiment2.csv};
                \end{axis}
            \end{tikzpicture}



        \subsection{Third experiment}
            \centering    
            \begin{tikzpicture}
                \begin{axis}[
                        ylabel=$T$,
                        xlabel=$t$,
                        xmin=0, xmax=330,
                        ymin=22, ymax=22.2,
                        xtick={0,20,...,300},
                        xticklabel style={rotate=45, anchor=east},
                        ytick={22,22.02,...,22.2},
                        axis lines=middle,
                        grid=both
                    ]
                    \addplot [only marks] table [col sep=comma] {../data/raw/experiment3.csv};
                \end{axis}
            \end{tikzpicture}

    \section{Conclusions}
        bla-bla-bla















    
    \newpage
    \section{References}
        1
        2
        3
        4

\end{document}